\documentstyle{article}
\begin{document}

\section{Pré-calculo}
Antes de qualquer O livro começa com uma sẽrie de exercicios considerados que eu considero "pré-calculo", abaixo segue as minhas soluções:



\subsection{Avalie se as expressões sem calculadora}

	\paragraph{a.}
	\begin{equation}
	(-3)^4 = (-3)\cdot(-3)\cdot(-3)\cdot(-3) = +81,
	\end{equation}
	 isto é, o valor que se encontra entre os parenteses multiplicado por ele mesmo quatro vezes.

	\paragraph{b.}
	\begin{equation}
	-3^4 = -(3^4) = -(3\cdot3\cdot3\cdot3) = -81,
	\end{equation}
	 nesse caso o numeral que não esta entre parentes é submetido as operações de potenciação e só então recebe o sinal da expressão, tendo como resultado um número negativo.

	\paragraph{c.}
	\begin{equation}
	3^{-4} = \frac{1}{3^4} = \frac{1}{81},
	\end{equation}
	 de forma que a primeira substituição que podemos fazer é a da propriedade da potenciação que afirma que $a^-1 = \frac{1}{a}$.

	\paragraph{d.}
	\begin{equation}
	\frac{5^{23}}{5^{21}} = 5^2 = 25,
	\end{equation}
	Podemos afirmar esse resultado por conta de uma propriedade da potenciação que define que $\frac{a^x}{a^y} = a^{x-y}$.

	\paragraph{e.}
	\begin{equation}
	\left(\frac{2}{3}\right)^{-2} = \left(\frac{3}{2}\right)^{2} = \left(\frac{3^2}{2^4}\right) = \frac{9}{4},
	\end{equation}
	Usando a mesma propriedade do item C nos podemos simplicaficar a potencia negativa invertendo a base, e depois disso realizar as potenciações independentes de cada numeral.

	\paragraph{f.}
	\begin{equation}
	16^{-\frac{3}{4}} = \frac{1}{16^{\frac{3}{4}}} = \frac{1}{\sqrt[4]{16^3}} = \frac{1}{\sqrt[4]{2^{4^{3}}}} = \frac{1}{2^3} = \frac{1}{8} 
	\end{equation}
	Esse foi o exercicio mais chatinho dessa verificação, ele fez questão de lembrar como a atenção é um importante elemento do estudo matematico, tive dificuldades ao resolver $\sqrt[4]{16^3}$ por falta de atenção na hora de fatorar os números necessarios.




%\section{...}
%\subsection{...}
%\subsubsection{...}
%\paragraph{...}
%\subparagraph{...}

\end{document}
