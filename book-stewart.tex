\documentclass{article}

\usepackage{amsmath}
\usepackage{amssymb}

\begin{document}

\section{Pré-calculo}
Antes de qualquer O livro começa com uma sẽrie de exercicios considerados que eu considero "pré-calculo", abaixo segue as minhas soluções:



\subsection{Avalie se as expressões sem calculadora}

	\paragraph{a.}
	\begin{equation}
	(-3)^4 = (-3)\cdot(-3)\cdot(-3)\cdot(-3) = +81,
	\end{equation}
	 isto é, o valor que se encontra entre os parenteses multiplicado por ele mesmo quatro vezes.

	\paragraph{b.}
	\begin{equation}
	-3^4 = -(3^4) = -(3\cdot3\cdot3\cdot3) = -81,
	\end{equation}
	 nesse caso o numeral que não esta entre parentes é submetido as operações de potenciação e só então recebe o sinal da expressão, tendo como resultado um número negativo.

	\paragraph{c.}
	\begin{equation}
	3^{-4} = \frac{1}{3^4} = \frac{1}{81},
	\end{equation}
	 de forma que a primeira substituição que podemos fazer é a da propriedade da potenciação que afirma que $a^-1 = \frac{1}{a}$.

	\paragraph{d.}
	\begin{equation}
	\frac{5^{23}}{5^{21}} = 5^2 = 25,
	\end{equation}
	Podemos afirmar esse resultado por conta de uma propriedade da potenciação que define que $\frac{a^x}{a^y} = a^{x-y}$.

	\paragraph{e.}
	\begin{equation}
	\left(\frac{2}{3}\right)^{-2} = \left(\frac{3}{2}\right)^{2} = \left(\frac{3^2}{2^4}\right) = \frac{9}{4},
	\end{equation}
	Usando a mesma propriedade do item C nos podemos simplicaficar a potencia negativa invertendo a base, e depois disso realizar as potenciações independentes de cada numeral.

	\paragraph{f.}
	\begin{equation}
	16^{-\frac{3}{4}} = \frac{1}{16^{\frac{3}{4}}} = \frac{1}{\sqrt[4]{16^3}} = \frac{1}{\sqrt[4]{2^{4^{3}}}} = \frac{1}{2^3} = \frac{1}{8} 
	\end{equation}
	Esse foi o exercicio mais chatinho dessa verificação, ele fez questão de lembrar como a atenção é um importante elemento do estudo matematico, tive dificuldades ao resolver $\sqrt[4]{16^3}$ por falta de atenção na hora de fatorar os números necessarios.

\subsection{Simplifique as expressões, escrevendo-as sem expoentes negativos}

	\paragraph{a.}
	\begin{equation}
	\sqrt{200} - \sqrt{32} = 10\sqrt{2} - 4\sqrt{2} = 5\sqrt{2}
	\end{equation},
	Podemos simplificar as raizes fatorando os números até que eles sejam da mesma grandeza, permitindo assim a soma simples.

	\paragraph{b.}
	\begin{equation}
	(3a^3b^3)\cdot(4ab^2)^2 = (3a^3b^3)\cdot(16a^2b^4)= 48a^5b^7
	\end{equation},
	nesse exemplo nos podemos usar as propriedadas da multiplicação de potencias para facilitar a distributiva da expressão.
	
	\paragraph{c.}
	\begin{equation}
	\left(\frac{3x^{\frac{3}{2}}y^3}{x^2y^{-\frac{1}{2}}}\right)^{-2} = \frac{3^{-2}x^{-3}y^{-6}}{x^{-4}y} = \frac{x}{9y^7},
	\end{equation}
	Como de costumo, o exercicio final apresenta maior dificuldade, mas com uma atenção aos detalhes muita coisa nessa expressão pode ser simplificada facilitando sua resolução.


\subsection{Expanda e simplifique}
	
	\paragraph{a.}
	\begin{equation}
	3(x+6)+4(2x-5) = 11x - 2,
	\end{equation}
	pode ser resolvida facilmente aplicando a distributiva pelos fatores da expressão.

	\paragraph{b.}
	\begin{equation}
	(x+3)\cdot(4x-5) = 4x^2 + 7x - 15
	\end{equation}
	exatamente igual ao exercicio anterior essa pode ser facilmente resolvida com uma distribuiva.

	\paragraph{c.}
	\begin{equation}
	(\sqrt{a} + \sqrt{b})\cdot(\sqrt{a} - \sqrt{b}) = a -b
	\end{equation}
	Esse pode ser resolvido de cabeça pois nos relembra um produto notavel, e o uso desse produto é bem comum por toda matematica.

	\paragraph{d.}
	\begin{equation}
	(2x+3)^2 = 4x^2+12x+9
	\end{equation}
	Esse é o exemplo mais comum de produtos notaveis, não podemos nos esquecer da regra `o quadrado do primeiro, duas vezes o produto do primeiro pelo segundo, o quadrado do segundo`. Pode-se, obviamente, aplicar a distributiva nessa expressão, mas o fato dela ser um produto notavel torna desnecessario tal processo.

	\paragraph{e.}
	\begin{equation}
	(2x+3)^3 = (2x+3) \cdot (2x+3) \cdot (2x+3) = (4x^2 + 12x + 9) \cdot (2x+3) = 8x^3 + 36x^2 + 36x + 18x + 27
	\end{equation}
	Resultando em um polinomio de terceiro grau, esse exercicio já tinha sido começado a ser feito no enunciado anterior, sabemos que sabemos que esse polinomio tem 3 soluções iguais tais que suas raizes é $x = -\frac{3}{2}$.


\subsection{Fatore cada expressão}
	\paragraph{a.}
	\begin{equation}
		4x^2 - 25 = 0 \therefore 4x^2 = 25 \therefore x= \frac{\sqrt{25}}{\sqrt{4}} \therefore x= \frac{5}{2},
	\end{equation}
	A unica dificuldade real desse problema é a de que só ao se tirar a raiz de ambos os lados o numeral $4$ deve ser levado em conta, com isso em mente o exercicio se torna trivial.

	\paragraph{b.}
	\begin{equation}
		
	\end{equation}

	














%\section{...}
%\subsection{...}
%\subsubsection{...}
%\paragraph{...}
%\subparagraph{...}

\end{document}
