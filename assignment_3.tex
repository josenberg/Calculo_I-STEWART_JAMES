%----------------------------------------------------------------------------------------
%	PACKAGES AND OTHER DOCUMENT CONFIGURATIONS
%----------------------------------------------------------------------------------------

\documentclass[paper=a4, fontsize=11pt]{scrartcl} % A4 paper and 11pt font size

\usepackage[T1]{fontenc} % Use 8-bit encoding that has 256 glyphs
\usepackage{fourier} % Use the Adobe Utopia font for the document - comment this line to return to the LaTeX default
\usepackage[brazilian]{babel}
\usepackage[utf8]{inputenc}
\usepackage[T1]{fontenc}
\usepackage{amsmath,amsfonts,amsthm} % Math packages
\usepackage{pgfplots}



\usepackage{lipsum} % Used for inserting dummy 'Lorem ipsum' text into the template

\usepackage{sectsty} % Allows customizing section commands
\allsectionsfont{\centering \normalfont\scshape} % Make all sections centered, the default font and small caps

\usepackage{fancyhdr} % Custom headers and footers
\pagestyle{fancyplain} % Makes all pages in the document conform to the custom headers and footers
\fancyhead{} % No page header - if you want one, create it in the same way as the footers below
\fancyfoot[L]{} % Empty left footer
\fancyfoot[C]{} % Empty center footer
\fancyfoot[R]{\thepage} % Page numbering for right footer
\renewcommand{\headrulewidth}{0pt} % Remove header underlines
\renewcommand{\footrulewidth}{0pt} % Remove footer underlines
\setlength{\headheight}{13.6pt} % Customize the height of the header

\numberwithin{equation}{section} % Number equations within sections (i.e. 1.1, 1.2, 2.1, 2.2 instead of 1, 2, 3, 4)
\numberwithin{figure}{section} % Number figures within sections (i.e. 1.1, 1.2, 2.1, 2.2 instead of 1, 2, 3, 4)
\numberwithin{table}{section} % Number tables within sections (i.e. 1.1, 1.2, 2.1, 2.2 instead of 1, 2, 3, 4)

\setlength\parindent{0pt} % Removes all indentation from paragraphs - comment this line for an assignment with lots of text

%----------------------------------------------------------------------------------------
%	TITLE SECTION
%----------------------------------------------------------------------------------------

\newcommand{\horrule}[1]{\rule{\linewidth}{#1}} % Create horizontal rule command with 1 argument of height

\title{	
\normalfont \normalsize 
\textsc{Americana, josenberg.com} \\ [25pt] % Your university, school and/or department name(s)
\horrule{0.5pt} \\[0.4cm] % Thin top horizontal rule
\huge Cálculo, V.1 - Stewart, James  \\ % The assignment title
\horrule{2pt} \\[0.5cm] % Thick bottom horizontal rule
}

\author{Philipe Godoy} % Your name

\date{\normalsize\today} % Today's date or a custom date

\begin{document}

\maketitle % Print the title

%----------------------------------------------------------------------------------------
%	PROBLEM 1
%----------------------------------------------------------------------------------------

\section{Pré-calculo}

\lipsum[2] % Dummy text

\begin{align} 
\begin{split}
(x+y)^3 	&= (x+y)^2(x+y)\\
&=(x^2+2xy+y^2)(x+y)\\
&=(x^3+2x^2y+xy^2) + (x^2y+2xy^2+y^3)\\
&=x^3+3x^2y+3xy^2+y^3
\end{split}					
\end{align}

Antes de qualquer O livro começa com uma série de exercicios considerados que eu considero "pré-calculo", abaixo segue as minhas soluções:


\section{Capitulo 0 - Para que serve calculo?}
Na grecia antiga, os matematicos já sabiam como calcular a area de qualquer figura geometrica dividindo ela em triangulos, dessa forma eles já conseguiam calcular a área da maioria das figuras, exceto a da circunferencia, porém é possivel notar que é possivel colocar uma regular circunscrita na circunferencia afim de estimar sua área, e quanto mais lados tiver essa figura mais proximo a area da circunferencia podemos chegar. Uma figura geometrica regular com lados infinitos circunscrita na circunferencia tem a mesma area que ela, a pergunta ficou "Como podemos calcular a área de uma figura com infinitos lados?". 

Um problema bem parecido aconteceu quando os matematicos tentavam calcular a área que determinada função fazia sobre o grafico,
a maneira mais adequada de fazer isso, seria encontrar a tendencia de variação entre os coeficientes angulares que os pontos possuiam e ao calcular essa diferença de infinitos pontos eles poderiam encontrar 

Esse é um dos focos do calculo, para estimar uma grandeza nos somamos uma infinidade de numeros pertecentes a uma sequencia e obtemos um numero muito proximo da que seria o ideal.

Um dos exemplos apresentados no primeiro capitulo é a soma de 

\begin{equation}
\left(\frac{1}{3}\right) = \left(\frac{3}{10}\right) + \left(\frac{3}{100}\right) + \left(\frac{3}{1000}\right) + \cdot\cdot\cdot
\end{equation}

Podemos assumir que ao somamos infinitas frações nos chegaremos à $\frac{1}{3}$ esse tipo de processo é chamado de limite, e é escrito como: 

\begin{equation}
\lim_{x \to \infty} \frac{3}{10^n} = \frac{1}{3}
\end{equation}

O calculo é usado em todos os tipos de ciencia para explicar fenomenos e comportamentos, assim os estimando e os entendendo.

\section{Capitulo I - Funções e modelos}

Uma função pode ser representada de diversas maneiras - um gráfico, equação, tabela ou até mesmo por meio de palavas.
Uma função $ f $ é uma lei que associa cada elemento de um conjunto $ D $ a exatamente um unico outro elemento de um conjunto$ E $.
Conjunto $ E $ é chamado de dominio da função, é o conjunto de todos os valores que podem assumir $ f(x) $ quando variamos a variavel X por todo o dominio.

É possivel considerar a função como uma maquina tal que:

\begin{equation}
x (entrada) \rightarrow F (processamento) \rightarrow f(x) (saída)
\end{equation}

Funções com finitos elementos também podem ser representadas com um diagrama de flexas.

O metodo mais comum de se representar uma fnução é a partir de uma equação que define define seu grafico, essas equações seguem a o formato:

\begin{equation}
{(x, f(x)) | x \in D}
\end{equation}

\clearpage

O livro dá alguns exemplos de funções e pede para que esbolsemos o gráfico que as reprenseta, como tenho um computador a disposição vou desenha-las para mais fácil entendimento. Segue os gráficos abaixo:

\begin{tikzpicture}
 \begin{axis}[    
   axis lines = middle, 
   axis line style={->},  
   xlabel=$x$,
   ylabel={$f(x) = 2x - 1$}
 ] 
   \addplot [black, domain=-5:5]{2*x - 1}; 
 \end{axis}
\end{tikzpicture}

\begin{tikzpicture}[trim axis left]
 \begin{axis}[    
   axis lines = middle, 
   axis line style={->},  
   xlabel=$x$,
   ylabel={$f(x) = x^2$}
 ] 
   \addplot [black, domain=-5:5]{x^2}; 
 \end{axis}
\end{tikzpicture}

Um dos exercicios ele pede para encontrarmos o dominio da função $g(x) = \frac{1}{x^2}$, para fazer isso precisamos garantir que o divisor nunca seja nulo, dessa forma $x^2 -x \ne 0$, desenvolvemos a equação de modo que: 

\begin{equation}
x2-x \ne 0
\end{equation}







%------------------------------------------------

%\subsection{Heading on level 2 (subsection)}

%Lorem ipsum dolor sit amet, consectetuer adipiscing elit. 
%\begin{align}
%A = 
%\begin{bmatrix}
%A_{11} & A_{21} \\
%A_{21} & A_{22}
%\end{bmatrix}
%\end{align}
%Aenean commodo ligula eget dolor. Aenean massa. Cum sociis natoque penatibus et magnis dis parturient montes, nascetur ridiculus mus. Donec quam felis, ultricies nec, pellentesque eu, pretium quis, sem.

%------------------------------------------------

%\subsubsection{Heading on level 3 (subsubsection)}

%\lipsum[3] % Dummy text

%\paragraph{Heading on level 4 (paragraph)}

%\lipsum[6] % Dummy text

%----------------------------------------------------------------------------------------
%	PROBLEM 2
%----------------------------------------------------------------------------------------

%\section{Lists}

%------------------------------------------------

%\subsection{Example of list (3*itemize)}
%\begin{itemize}
%	\item First item in a list 
%		\begin{itemize}
%		\item First item in a list 
%			\begin{itemize}
%			\item First item in a list 
%			\item Second item in a list 
%			\end{itemize}
%		\item Second item in a list 
%		\end{itemize}
%	\item Second item in a list 
%\end{itemize}

%------------------------------------------------

%\subsection{Example of list (enumerate)}
%\begin{enumerate}
%\item First item in a list 
%\item Second item in a list 
%\item Third item in a list
%\end{enumerate}

%----------------------------------------------------------------------------------------

\end{document}